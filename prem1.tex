%---------------------------------------------------------
%  Cap 1 - esqueleto 
%---------------------------------------------------------

\documentclass[a4paper,12pt]{report}

\usepackage{amsfonts}
\usepackage{amsthm}
\usepackage{amsmath}
\usepackage{latexsym}
\usepackage{amssymb}
\usepackage{graphicx}
%\usepackage{dsfont}
%\usepackage[utf8]{inputenc}
\usepackage[T1]{fontenc}
\usepackage[spanish]{babel}


\usepackage{amsfonts, amsthm, amsmath, amsxtra, amstext,amssymb, latexsym} % fuentes

%\usepackage{pxfonts} %%% le pone una fuente curvita
\usepackage{imakeidx}
\makeindex

\usepackage{parskip}
\setlength{\parskip}{1.3ex plus 0.2ex minus 0.2ex} % linea entre parrafos
\setlength{\parindent}{1em} % sangria


\usepackage[
 pagebackref,
 bookmarksnumbered=true,
 breaklinks=true,
 pdfborder=0 0 0
]{hyperref}  % backref linktocpage pagebackref
\pdfcompresslevel=9
\pdfadjustspacing=1 
\PassOptionsToPackage{dvipsnames}{xcolor}
        \RequirePackage{xcolor} % [dvipsnames] 
\definecolor{Maroon}{cmyk}{0, 0.87, 0.68, 0.32}
\definecolor{halfgray}{gray}{0.55}
\definecolor{webbrown}{rgb}{.6,0,0}
\definecolor{webgreen}{rgb}{0,.5,0}
\definecolor{RoyalBlue}{cmyk}{1, 0.50, 0, 0}
\hypersetup{%
  %draft,     % = no hyperlinking at all (useful in b/w printouts)
  colorlinks=true, linktocpage=true, pdfstartpage=2, pdfstartview=FitV,%
  % uncomment the following line if you want to have black links (e.g., for printing)
  %colorlinks=false, linktocpage=false, pdfborder={0 0 0}, pdfstartpage=3, pdfstartview=FitV,% 
  breaklinks=true, pdfpagemode=UseNone, pageanchor=true, pdfpagemode=UseOutlines,%
  plainpages=false, bookmarksnumbered, bookmarksopen=true, bookmarksopenlevel=1,%
  hypertexnames=true, pdfhighlight=/O,%nesting=true,%frenchlinks,%
  urlcolor=webbrown, linkcolor=RoyalBlue, citecolor=webgreen, %pagecolor=RoyalBlue,%
  %urlcolor=Black, linkcolor=Black, citecolor=Black, %pagecolor=Black,%
%  pdftitle={\myTitle},%
%  pdfauthor={\textcopyright, \myName, \myUni, \myFaculty},%
  pdfsubject={},%
  pdfkeywords={},%
  pdfcreator={pdfLaTeX},%
  pdfproducer={LaTeX with 'Venia latexando como un campeon'}%
}
\newcommand{\backrefnotcitedstring}{\relax}%(Not cited.)
\renewcommand*{\backref}[1]{}  % disable standard
\renewcommand*{\backrefalt}[4]{% detailed backref
  \ifcase #1 %
  \backrefnotcitedstring%
  \or%
  (citado en p\'agina #2)%
  \else%
  (citado en p\'aginas #2)%
  \fi}%
\makeatletter
\let\stdl@section\l@section
\renewcommand*{\l@section}[2]{%
  \stdl@section{\textcolor{black}{#1}}{\textcolor{blue}{#2}}}
\makeatother

\usepackage{epsfig}   % añadir figuras en eps
\usepackage{graphicx} % añadir JPG, PDF, PNG compilando en pdf
\usepackage{enumerate}% Para cambiar el tipo de enumeración de las
                      % listas
\usepackage{fancyhdr} % Estilo de página
\usepackage{lastpage} % Para usar el comando que nos devuelve el
                      % número de la última página, útil para numerar
                      % las páginas (ej: página 1 de 5)
\usepackage{cancel}   % para.. bueno, eso. 
%\usepackage{simpsons}

\pagestyle{fancy} % Estilo de las páginas



\newcommand{\toclesssection}[1]{\section*{#1}\addtocounter{section}{1}}
%esto es para no manchar toda la ropa al lavar la media roja

\renewcommand{\subsectionmark}[1]{\markboth{#1}{#1}}
\renewcommand{\sectionmark}[1]{\markboth{#1}{#1}} % Necesario para que
                                                  % el \leftmark
                                                  % devuelva el título
                                                  % de la sección pero
                                                  % no el número

%%%\renewcommand{\headrulewidth}{0.4pt}% Grosor de la linea de la
                                    % cabecera, si el valor es 0 la
                                    % linea desaparece
%%%\renewcommand{\footrulewidth}{0.4pt}% Grosor de la linea del pie, si
                                    % el valor es 0 la linea
                                    % desaparece

% Cabecera y pie de página
% Páginas pares
\fancyhead[R]{\leftmark}
\fancyhead[C]{}
\fancyhead[L]{Tesis de Doctorado}

\fancyfoot[R]{\thepage}
\fancyfoot[C]{}
\fancyfoot[L]{Departamento de Matemática - UNLP}


% Ahora definimos los Teoremas, definiciones, corolarios, lemas y
% demás, lo que está entre llaves es la abreviatura y lo que está
% entre corchetes es la numeración que deben de seguir, el * implica
% que no se numera.----- chusmear cual se puede volar


% THEOREMS -------------------------------------------------------
\newtheorem{fed}{Definición}[section]
\newtheorem{teo}[fed]{Teorema}
\newtheorem{lem}[fed]{Lema}
\newtheorem{cor}[fed]{Corolario}
\newtheorem{pro}[fed]{Proposición}

\theoremstyle{definition}
\newtheorem{rem}[fed]{Observación}
\newtheorem{rems}[fed]{Remarks}
\newtheorem{exa}[fed]{Ejemplo}
\newtheorem{exas}[fed]{Examples}
\newtheorem{num}[fed]{}

\newcommand{\norm}[1]{\left|\left| {#1} \right|\right|}
%\newcommand{\dnorm}[1]{\left|\left|\left|\left {#1} \right|\right \right|\right|}
\newcommand{\prin}[1]{\langle {#1} \rangle}
\newcommand{\modu}[1] { \left| {#1} \right| }
\newcommand{\tr}[1]{\textnormal{tr}\left({#1}\right)}




\renewcommand{\eqref}[1]{ \textnormal{(\ref{#1})} }
\renewcommand{\qed}{\hfill \ensuremath{\blacksquare}}
%\renewenvironment{proof}{{\bfseries Demostraci\'on}}-intento.fallido


% ----------------------------------------------------------------
% ----------------------------------------------------------------
% ----------------------------------------------------------------

\def\inv{^{-1}}

\def\cA{{\cal A}}
\def\cP{{\cal P}}
\def\cQ{{\cal Q}}
\def\cS{{\cal S}}
\def\cT{{\cal T}}
\def\cM{{\cal M}}
\def\cN{{\cal N}}
\def\M{{\mathcal M}}
\def\N{{\mathcal N}}

\def\cac{{\cal A}_{a\mrai ba\mrai}}

\def\cA{\mathcal{A}}
\def\cB{\mathcal{B}}
\def\cC{\mathcal{C}}
\def\C{\mathcal{C}}
\def\ede{\mathcal{D}}
\def\edemas{\mathcal{D}^+}
\def\edep{\mathcal{P}(\mathcal{D})}
\def\cE{\mathcal{E}}
\def\cH{\mathcal{H}}
\def\cK{\mathcal{K}}
\def\ccK{\mathcal{K}}
\def\cU{\mathcal{U}}
\def\ele{\mathcal{L}}
\def\cP{\mathcal{P}}
\def\cQ{\mathcal{Q}}


\def\cO{{\mathcal O}}

\def\R{\mathbb{R}}

\def\ese{\mathcal{S}}
\def\ete{\mathcal{T}}
\def\eme{\mathcal{M}}
\def\ene{\mathcal{N}}
\def\cW{\mathcal{W}}
\def\cX{\mathcal{X}}
\def\cY{\mathcal{Y}}
\def\cZ{\mathcal{Z}}
\def\casa{\mathcal{A}_{sa}}
\def\capo{\mathcal{A}_{+}}



\def\inc{\subseteq}

\def\suml{\sum\limits}


\def\uno{\mbox{\textbf{e}}}


\def\cx{C(X)}
\def\cstar{$C^*$-algebra $\;$}
\def\cstars{$C^*$-algebras $\;$}



\def\bp{\veebar}
\def\TT{{\rm T} \hskip -5pt{\rm T} \hskip4pt}
\def\bull{\vrule height 1.0ex width .4ex depth -.1ex }
\def\vacio{\emptyset}
\def\orto{^\perp}
\def\inc{\subseteq}
\def\subcer{\sqsubseteq}


%\def\QED{\hfill \bull}
\def\sii{ if and only if }
\def\inv{^{-1}}
\def\*A{\#\sb A}
\def\pa{P}
\def\eps{\varepsilon}
\def\fii{\varphi }
\def\fia{\varphi _a}
\def\fib{\varphi _b}
\def\uni{{\cal U}(\H )}
\def\H{{\cal H}}

%\def\unia{{\cal U}_a}
%\def\uniA{{\cal U}_A}
%\def\unib{{\cal U}_b}
\def\Gpo{Gl(\H)^+}
\def\glh{Gl(\cH)}
\def\ca{L(\H ) }
\def\cam{L(\H )^+ }
\def\As{\ca _s}
\def\Aq{\ca _q}
\def\cH{{\cal H}}
\def\dem{\paragraph{Proof.}}
\def\kerp{\cQ^P}


\def\PA{P_{A, \cS}}
\def\PAn{P_{A, \cS_n}}
\def\PB{P_{B, \cS}}
\def\QA{Q_{A,\cS} }
\def\csha{\Sigma (P, A)}
\def\PAS{\cP(A, \cS)}
\def\PBS{\cP(B, \cS)}
\def\str{^{\#}}
\def\spline{sp \ (T, \cS , \xi )}
\def\splineA{sp \ (A\rai, \cS , \xi )}
\def\rai{^{1/2}}
\def\mrai{^{-1/2}}
\def\pim{\pi^+}
\def\piA{\pi_A}
\def\api{\langle}
\def\cpi{\rangle}
\def\pou{\Theta}
\def\noi{\noindent}
\def\qa{\cQ}
\def\csta{C$^*$-algebra}
\def\upi{\Upsilon}
\def\bm{\left(\begin{array}}
\def\em{\end{array}\right)}
\def\Gs{G^s}
\def\ben{\begin{enumerate}}
\def\een{\end{enumerate}}
%\def\begin{equation}{\begin {equation}}
%\def\end{equation}{\end {equation}}
\def\barr{\begin{array}}
\def\earr{\end{array}}
\def\iiff{if and only if }
\def\inv{^{-1}}
\def\pa{\cP}
\def\auta{{\cal S} \sb a }
\def\U{\cal U (\H )}
\def\H{{\cal H}}
%\def\glh{{GL(\H)}}
%\def\g+{{\glh^+}}
\def\lh{{L(\H)}}
\def\lh+{{\lh^+}}
\def\cas{{\ca_s}}
\def\gs{{\glh_s}}
%\def\der{Der(\M)}
\def\pia{\pi\sb \alpha}
\def\za{{\cal Z} (\A1)}
\def\la{\lambda}
\def\eps{\varepsilon}
\def\cX{{\cal X}}
\def\cY{{\cal Y}}
\def\cZ{{\cal Z}}
\def\com{$(A, \cS)$ is compatible}
\def\compn{$(A, \cS_n)$ is compatible}
\def\comB{$(B, \cS)$ is compatible}
\def\EOE{\hfill$\triangle$}



\newcommand{\corch}[1]{\left[ #1 \right]}

\newcommand{\trivial}{\{0\}}


%\renewcommand{\qed}{\hfill$\blacksquare$}





% ----------------------------------------------------------------
% ----------------------------------------------------------------
% Math operators
% ----------------------------------------------------------------
% ----------------------------------------------------------------

\DeclareMathOperator{\Preal}{Re} \DeclareMathOperator{\Pim}{Im}
\DeclareMathOperator*{\limsot}{\lim}
\DeclareMathOperator*{\convsotdpre}{\searrow}
\DeclareMathOperator*{\convsotipre}{\nearrow}
\DeclareMathOperator{\Tr}{Tr} 
%\DeclareMathOperator{\tr}{tr}
\DeclareMathOperator*{\truchada}{=}
\DeclareMathOperator*{\inversible}{GL}

\DeclareMathOperator{\leqt}{\preceq}
\DeclareMathOperator{\geqt}{\succeq}
\DeclareMathOperator{\leqrt}{\precsim}
\DeclareMathOperator{\geqrt}{\succsim}
\DeclareMathOperator{\leqsm}{\prec}
\DeclareMathOperator{\geqsm}{\succ}
\DeclareMathOperator{\leqm}{\preccurlyeq}
\DeclareMathOperator{\geqm}{\succcurlyeq}
\DeclareMathOperator{\cc}{\mbox{cc}}




\newcommand{\pint}[2]{\displaystyle \left \langle #1,#2 \right\rangle}

\newcommand{\hil}{\mathcal{H}}
\newcommand{\op}{L(\mathcal{H})}
\newcommand{\opsa}{L_{sa}(\mathcal{H})}
\newcommand{\posop}{L(\mathcal{H})^+}
\newcommand{\gop}{GL(\mathcal{H})}
\newcommand{\gposop}{GL(\mathcal{H})^+}

\newcommand{\bon}{\{e_k\}_{k\in \mathbb{N}}}
\newcommand{\bonn}{\{e_n\}_{n\in \mathbb{N}}}
\newcommand{\primerosn}{\{e_k\}_{1\leq k\leq n}}
\newcommand{\briesz}{\{x_k\}_{k\geq 1}}
\newcommand{\Dpint}[1]{\displaystyle \left \langle #1 \right\rangle_D}
\newcommand{\generado}[1]{\mbox{span}\left\{#1\right\}}


 

% ----------------------------------------------------------------
% ----------------- convergencias
% ----------------------------------------------------------------

\newcommand{\conv}{\xrightarrow[n\rightarrow\infty]{}}
\newcommand{\convnorm}{\xrightarrow[n\rightarrow\infty]{\|\,\cdot\,\|}}
\newcommand{\convnormd}{\convsotdpre_{n\rightarrow\infty}^{\|\,\cdot\,\|}}
\newcommand{\convnormi}{\convsotipre_{n\rightarrow\infty}^{\|\,\cdot\,\|}}

\newcommand{\convsot}{\xrightarrow[n\rightarrow\infty]{\mbox{\tiny{S.O.T.}}}}
\newcommand{\convsotd}{\convsotdpre_{n\rightarrow\infty}^{\mbox{\tiny
\textbf SOT}}}
\newcommand{\convsoti}{\convsotipre_{n\rightarrow\infty}^{\mbox{\tiny
\textbf SOT}}}

\newcommand{\convwotd}{\convsotdpre_{n\rightarrow\infty}^{\mbox{\tiny
\textbf WOT}}}
\newcommand{\convwoti}{\convsotipre_{n\rightarrow\infty}^{\mbox{\tiny
\textbf WOT}}}
\newcommand{\convwot}{\xrightarrow[n\rightarrow\infty]{\mbox{\tiny{W.O.T.}}}}

\newcommand{\convctp}{\xrightarrow[n\rightarrow\infty]{\mbox{\tiny{C.T.P.}}}}

\newcommand{\nui}[1]{||| #1 |||}
\newcommand{\ninf}[1]{\left\| #1 \right\|_{\infty}}

\newcommand{\espro}[2]{E_{#1}\left[#2\right]}

%\newcommand{\op}{L(\mathcal{H})}

% ----------------------------------------------------------------
% -------------- familias de marcos de vectores y de fusion
% ----------------------------------------------------------------


\def\ssec{\cW = \{ W_i\}_{i\in I}}
\def\sseck{\cW = \{ W_k\}_{k\in \NN}}
\def\ssecV{\cV = \{ V_i\}_{i\in I}}
\def\refi{\cV = \{ V_i\}_{i\in J}}


\def\sfram{\cW_w  = (w_i\, ,\, W_i)_{i\in I}}
\def\sframJ{ (w_i\, ,\, W_i)_{i\in J}}
\def\sframN{\cW_w = (w_k\, ,\, W_k)_{k\in \NN}}
\def\FS {\cW_w\,}
\def\sfram{\cW_w  = (w_i\, ,\, W_i)_{i\in I}}
\def\sframJ{ (w_i\, ,\, W_i)_{i\in J}}
\def\sframN{\cW_w = (w_k\, ,\, W_k)_{k\in \NN}}

\def \linm{\ell^\infty _+ (I)}
\def \linmJ{\ell^\infty _+ (J)}
\def \linmN{\ell^\infty _+ (\NN)}

\newcommand{\exe}[1]{E \left(#1\right)}

%operador de sintesis
\newcommand{\tfs}{T_{\cW_w}}

\def\ra{\rightarrow}

%\newcommand{\peso}[1]{ \quad \text{ #1 } \quad }

\newcommand{\fram}{\{f_n\}_{n\in \mathbb{N}}}
% \newcommand{\sfram}{\{f_i\}_{i\in I}}
\newcommand{\sframn}{\{f_i\}_{i\in I_n}}



\newcommand{\llav}[1]{\{#1\}}



% ----------------------------------------------------------------
% -------------- coqueteria del tex
% ----------------------------------------------------------------


\def\ds{\displaystyle} 
\def\noi{\noindent}
\def\QED{\hfill $\blacksquare$}


\def\coma{\, , \, }
\def\pausa{\medskip\noi}

\newcommand{\peso}[1]{ \quad \text{ #1 } \quad }


\def\py{\peso{and}}



%\newcommand{\bon}{\{e_i\}_{i\in \mathbb{N}}}
% ----------------------------------------------------------------
% ----------------------------------------------------------------
% Math operators
% ----------------------------------------------------------------
% ----------------------------------------------------------------
\DeclareMathOperator{\ran}{R} \DeclareMathOperator{\nuc}{N}
\DeclareMathOperator{\dist}{dist}

% ----------------------------------------------------------------
% ----------------------------------------------------------------
% ángulos y producto interno
% ----------------------------------------------------------------
% ----------------------------------------------------------------

%\newcommand{\pint}[1]{\displaystyle \left \langle #1 \right\rangle}
%\newcommand{\hil}{\mathcal{H}}
\newcommand{\angd}[2]{c_0\left(\,#1,\,#2\,\right)}
\newcommand{\angf}[2]{c\left(\,#1,\,#2\,\right)}
%\newcommand{\generado}[1]{\mbox{span}\left\{#1\right\}}
% ----------------------------------------------------------------
% ----------------------------------------------------------------


% ----------------------------------------------------------------
% ----------------------------------------------------------------
% espacios de vectores y matrices 
% ----------------------------------------------------------------
% ----------------------------------------------------------------

\newcommand{\cene}{\mathbb{C}^n}
\newcommand{\cede}{\mathbb{C}^d}

\newcommand{\mat}{\mathcal{M}_d(\mathbb{C})}
\newcommand{\matn}{\mathcal{M}_n(\mathbb{C})}
\newcommand{\matreal}{\mathcal{M}_n(\mathbb{R})}
\newcommand{\matsa}{\mathcal{H}(n)}
\newcommand{\matsad}{\mathcal{H}(d)}
\newcommand{\matsai}{\mathcal{H}_\inter(n)}
\newcommand{\mattr}{\mathcal{TS}(n)}
\newcommand{\matu}{\mathcal{U}(n)}
\newcommand{\matud}{\mathcal{U}(d)}
\newcommand{\matper}{\mathcal{U_P}(n)}
\newcommand{\matpos}{\mat^+}
\newcommand{\defpos}{\cP(d)}
\newcommand{\matposn}{\matn^+}
\newcommand{\matnpos}{\matn^+}
\newcommand{\matinv}{\mathcal{G}\textit{l}\,(n)}

\newcommand{\matinvd}{\mathcal{G}\textit{l}\,(d)}
\def\gld{\matinvd^+}



\def\I_d{\mathbb{I}_d}


%%%% para la mayo

\def\ua{^\uparrow}
\def\da{^\downarrow}

\DeclareMathOperator{\leqp}{\leqslant}
\DeclareMathOperator{\geqp}{\geqslant}




